% !TeX root = ТеорияОтображений.tex

\chapter{Теория множеств}

\begin{commentary}
Первой темой является вводный курс по теории множеств, в котором затронутся базовые термины и понятия. Очень важно освоить этот раздел, поскольку на протяжении всего курса мы будем работать именно с множествами.
\end{commentary}

\section{Принадлежность}

Само множество является неопределяемым понятием, но мы все еще можем дать интуитивное представление, поэтому
\begin{definition}\label{def:set}
Множество -- это набор произвольных элементов. Элементы в множестве не упорядочены и каждый элемент учитывается только один раз \textit{(даже если случились повторения)}.
\end{definition}
Точнее говоря, множество определяется через набор своих свойств и действий, которые вообще можно делать с ними (например объединять). По научному мы выбираем набор аксиом, то есть утверждений, которые принимаются за истину и \textit{чаще всего} интуитивно понятны, а все остальное выводится с помощью логики. Классически используется аксиоматика Цермелло-Френкеля, мы не будем ее разбирать поскольку это выходит за рамки курса, но важно понимать, что множество -- объект на поверхности очень простой, но коварный, и халатное обращение с множествами приводит к парадоксам.

Теперь подумайте, а как вообще можно задать множество? Самый простой ответ -- список, т.е. явно перечислить элементы
$$M=\{a,b,c\}\ B=\{4,M,\{\delta\}\}\ Y=\{0,1,\{x,1\}\}$$
Классически само множество обозначается заглавной буквой латинского алфавита, a элементы перечисляются внутри фигурныйх скобок через запятую. Определение \ref{def:set} не запрещает одному множеству быть элементом другого, образуя как бы пакет с пакетами. Более того множество может лежать само в себе. Отдельно отметим, что во всех приведенных выше множествах ровно три элемента ($\{x,1\}$ считается как один элемент).

Любой элемент должен либо лежать в множестве, либо нет, третьего не дано, и самое важное что мы хотим уметь делать -- это проверять лежит ли конкретный элемент в данном множестве.
\begin{definition}
Утверждение 'Элемент b принадлежит множеству M' на математическом языке записывается как $b \in M$. Отрицание этого утверждения как $b \notin M$, то есть b не принадлежит M.
\end{definition}
И, к сожалению, уже это бывает сложной задачей, поэтому рассмотрим 
\begin{example}
В следующем множестве 5 элементов и каждый выделен отдельным цветом 
$$ A = \{ \textcolor{cyan}{\{a,b\}},\textcolor{red}{-8},\textcolor{phos}{\{0,\{0\}\}},\textcolor{blue}{\{0\}},\textcolor{magenta}{\Gamma} \} $$
и только про эти 5 наборов символов можно сказать, что они принадлежат множеству $A$. Записи $\{a\}\in A$ или $0\in A$ будут неверны, в отличии от $\{a,b\}\in A$, $-8\in A$, $\{0,\{0\}\}\in A$, $\{0\}\in A$ и $\Gamma\in A$. Утверждение $\{2\}\in \{1,\{1,2\}\}$ так же неверно.
\end{example}
Еще одним важным понятием в математике является пустое множество, оно обозначается как $\varnothing$ (или просто $\{\}$), и оно так же определяется через свои свойства, а именно
\begin{definition}\label{def:varnothing}
Существует множество $\varnothing$ такое, что для всякого элемента $a$ справедливо, что $a \notin \varnothing$.
\end{definition}
Фраза 'для всякого элемента $a\ldots$'\ часто требует пояснений. Она \textbf{не} означает что мы берем конкретно букву $a$ латинского алфавита. Все ровно наоборот, так обозначается произвольный элемент, который мы вообще можем себе представить. Конкретные элементы будут обозначаться буквой с индексом: $x_0,\,a_3,\,b_1\ldots$ Словосочетание 'для всякого' принято заменять символом '$\forall$', такие символы называются кванторами и нужны для более короткой записи математических утверждений. Определение \ref{def:varnothing} записывается в кванторах как
\[
\exists\varnothing: \forall a\ (a \notin \varnothing)
\]
Другие частые обозначения:
\begin{center}
'$\exists$' -- существует/найдется \quad ':' -- такой, что \quad '$:=$' -- равенство по определению. 
\end{center}
Из определения \ref{def:varnothing} сразу можно понять, что если вы видите запись $x_0 \in \varnothing$ \textit{(вставьте любой набор символов вместо $x_0$)}, то это либо противоречие, либо парадокс, поскольку она явно противоречит определению.
\begin{exercise}
Проверьте следующие утверждения
\begin{multicols}{3}
\noindent \textbf{1.}
$B = \{1,\{1\},\{1,2\}\}$\\
$1 \in B$\\
$2 \in B$\\
$\{1\} \in B$\\
$\{2\} \in B$\\
$\{1,\{1\}\} \in B$\\
$\{1,2\} \in B$
\bigskip\\
\noindent \textbf{2.}
$D = \{\{0,\{1\}\},\{\{0\},\{1\}\}\}$\\
$0 \in D$\\
$\{0\} \in D$\\
$\{0,\{1\}\} \in D$\\
$\{\{0\},1\} \in D$\\
$\{\{0\},\{1\}\} \in D$\\
$\{1\} \in D$
\columnbreak
\bigskip\\
\noindent \textbf{3.}
$F = \{0,5,\{0\},\{2\},\{\}\}$\\
$\{0\} \in F$\\
$\{0,5\} \in F$\\
$\{0,2\} \in F$\\
$\{0,\{0\}\} \in F$\\
$\varnothing \in F$\\
$\{\{\},\varnothing\} \in F$
\bigskip\\
\noindent \textbf{4.}
$C = \{\{1\},\{2\},\{1,2\}\}$\\
$1 \in C$\\
$2 \in C$\\
$\{1,1\} \in C$\\
$\{2\} \in C$\\
$\{1,2\} \in C$\\
$\{\{1\},\{2\}\} \in C$
\columnbreak
\bigskip\\
\noindent \textbf{5.}
$G = \{1, a, \{1,a\}, \{\{1\}\}\}$\\
$1 \in G$\\
$a \in G$\\
$\{a\} \in G$\\
$\{1\} \in G$\\
$\{1,a\} \in G$\\
$\{\{1,a\}\} \in G$
\bigskip\\
\noindent \textbf{6.}
Переведите на русский\\
$\forall n\in\mathbb{N}\ \exists m\colon m = 2n$\\
$\forall x\in\mathbb{Z}\ \exists y\in\mathbb{R}\colon y = \dfrac{1}{x}$\\
$\forall n\in\mathbb{Z}\ \exists k\in\mathbb{N}\colon k>n$\\
$\exists a\colon\forall n\in\mathbb{N}\quad \frac{1}{n}<a$\\
$\exists M\colon\forall n\in\mathbb{N}\ n<M$\\
$\exists c\in\mathbb{Z}\colon\forall n\in\mathbb{Z}\ n\le c$
\end{multicols}
\end{exercise}
\begin{commentary}
С помощью $\varnothing$ в теории множеств задаются натуральные числа, например число 4 будет определено следующим образом 
$$\{\varnothing,\{\varnothing\},\{\varnothing,\{\varnothing\}\},\{\varnothing,\{\varnothing\},\{\varnothing,\{\varnothing\}\}\}\}$$
или для большей аутентичности
$$\{\{\},\{\{\}\},\{\{\},\{\{\}\}\},\{\{\},\{\{\}\},\{\{\},\{\{\}\}\}\}\}$$
\end{commentary}
