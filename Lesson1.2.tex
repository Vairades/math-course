% !TeX root = ТеорияОтображений.tex

\section{Включение}
Помимо принадлежности элемента к множеству, возникает необходимость рассматривать множество не целиком, а только его часть. Отсюда естественным образом возникает 
\begin{definition}\label{def:subset}
Множество A включено в множество B, если:\\
для всякого $x \in A$ верно что $x \in B$.
Обозначается как $A \subset B$. Множество $A$ называют подмножеством множества $B$. В кванторах $A\subset B \overset{\mathrm{def}}{\iff}\forall x\ (x\in A\Rightarrow x\in B)$.

Неформальная формулировка звучит так: Все элементы первого множества должны быть элементами второго множества.

Полезно заранее выписать отрицание утверждения $A \subset B$, а именно найдется такой элемент $x_0 \in A$, что $x_0 \notin B$. Записывается оно как $A \not\subset B$.
\end{definition}
\begin{example}
Пусть $A=\{1,2,\{0\}\}$, проверим утверждение $\{1,2\} \subset A$. В множестве $\{1,2\}$ всего два элемента и достаточно явно проверить определение.
\begin{center}
$1 \in \{1,2\}$ и $1 \in A$ -- верно\\
$2 \in \{1,2\}$ и $2 \in A$ -- верно
\end{center}
Значит утверждение $\{1,2\} \subset A$ -- верно. Теперь проверим $\{0,2\} \subset A$ аналогичным образом
\begin{center}
$2 \in \{2,0\}$ и $2 \in A$ -- верно\\
$0 \in \{2,0\}$ и $0 \in A$ -- неверно
\end{center}
Поскольку $0 \notin A$ вся логическая цепочка рушится и получается что $\{0,2\} \not\subset A$.
\end{example}
\begin{statement}
Для всякого множества $A$ справедливо что $\varnothing \subset A$.
\end{statement}
\begin{proof}
На этом утверждении удобно продемонстрировать структуру математического доказательства, а также пару частых приемов. Прежде всего доказательство будет от противного, его структура выглядит так:\\
Предполагаем что утверждение 1.2.2 неверно $\rightarrow$ с помощью рассуждений получаем противоречие $\rightarrow$ мы предположили что 1.2.2 неверно и получили ошибку, значит 1.2.2 верно.

Предположим противное. Теперь необходимо сформулировать отрицание утверждения 1.2.2, оно звучит так
\begin{center}
Нашлось такое множество $A$, что $\varnothing \not\subset A$ -- это то, что мы предполагаем
\end{center}
По предположению $\varnothing \not\subset A$, распишем это утверждение -- найдется $x_1 \in \varnothing$, такой что $x_1 \notin A$ (см. \ref{def:subset}). Тот кто внимательно читал предыдущий раздел уже увидели, что фраза "найдется $x_1 \in \varnothing$"\,не может быть правдой, поскольку противоречит определению \ref{def:varnothing}. Мы пришли к противоречию (получили ложное утверждение), таким образом 1.2.2 верно.
\end{proof}
До сих пор упоминался только один способ задать множество -- явно перечислить его элементы. Этот способ удобен при работе с \textit{маленькими} множествами и абсолютно не подходит для всего остального. На практике сначала берется уже готовое множество, и из него выбирается подмножество, отвечающее конкретному условию. Классический шаблон, котором мы будем пользользоваться:
\[M =\textcolor{blue}{\{}\text{\textcolor{phos}{натуральные числа }}\textcolor{BrickRed}{\text{со свойством}}\, \textcolor{Purple}{\text{меньше }6}\textcolor{blue}{\}}\]
Или же на математическом
\[M =\textcolor{blue}{\{}\textcolor{phos}{x\in \mathbb{N}}\ \textcolor{BrickRed}{|}\ \textcolor{Purple}{x<6}\textcolor{blue}{\}}\]
Разберем подробнее. \textcolor{blue}{Фигурные скобки} обозначают границы множества, внутри описаны элементы, вещи снаружи нам не интересны. Далее мы видим вертикальный \textcolor{BrickRed}{разделитель} и текст по обе стороны. 

Текст \textcolor{phos}{слева} указывает что будет являться элементами множества \textit{(числа, формулы, функции, другие множества\ldots)}. Здесь можно указать из какого множества будут браться элементы \textbf{или} какое преобразование нужно сделать с элементом. Но очень важно чтобы до или после '\textcolor{BrickRed}{|}' было указано откуда берутся элементы. В нашем примере элементы из $\mathbb{N}$. Обозначение элемента $x$ выбрано произвольно, тоже множество можно задать и так
\[M =\textcolor{blue}{\{}\textcolor{phos}{\xi\in \mathbb{N}}\ \textcolor{BrickRed}{|}\ \textcolor{Purple}{\xi <6}\textcolor{blue}{\}}=\textcolor{blue}{\{}\textcolor{phos}{t\in \mathbb{N}}\ \textcolor{BrickRed}{|}\ \textcolor{Purple}{t<6}\textcolor{blue}{\}}=\{1,2,3,4,5\}\]
В математике важны не столько сами символы, сколько их связь между собой и взаимное расположение. 

Текст \textcolor{Purple}{справа} задает свойства, которым должны отвечать элемнты, в нашем примере элементы должны быть меньше 6. Сам \textcolor{BrickRed}{разделитель} заменяет слова 'со свойством' или 'такой, что'. Таким образом утверждение выше должно читаться так:
\begin{center}
$M$ -- множество элементов $\xi$ из натуральных чисел, со свойством $\xi$ меньше 6.
\end{center}
\begin{example}
\ \\
$B = \{\dfrac{h}{3}\mid h\in\mathbb{N},h<4\}=\{\dfrac{1}{3},\dfrac{2}{3},1\}$ -- тут $h$ делится на 3 перед попаданием в $B$\\[2pt]
$F=\{x^2\mid x \in \mathbb{N}\}$ -- здесь берутся натуральные числа и возводятся в квадрат\\
$I=\{x\in \mathbb{R}\ |\ x \notin \mathbb{Q}\}$ -- множество иррациональных чисел

Озвучим как читается последнее утверждение
\begin{center}
$I$ -- множество элементов $x$ из $\mathbb{R}$ таких, что $x$ не является рациональным.\\
Или множество элементов из $\mathbb{R}$, которые не рациональны.
\end{center}

Не лишним будет привести примеры некорректного задания множеств и частые ошибки\\
$M=\{\mathbb{Z}\ |\ x<6\}$ -- потерян элемент, не понятно что будет лежать в множестве\\
$K=\{x \in \mathbb{N}\ |\ y<7\}$ -- нет связи между названием элемента и условием. В множестве лежат иксы, а условие наложено на $y$.\\
$B=\{x\mid x>9\}$ -- не хватает информации о том, чем является икс. Целым? Рациональным? Это вообще число?\\
$D=\{a \subset \mathbb{N}\ |\ a<4\}$ -- перепутаны знаки $\in$ и $\subset$, из за чего смысл полностью потерян.\\
$Z=\{z\in\mathbb{N}\mid 2x\}$ -- справа нет условия.
\end{example}
\begin{multicols}{2}
\begin{exercise}
\ \\
\noindent \textbf{1.}
$C = \{ 1, 2, \{1,2\}, 0, \{r\} \}$\\
Найдите ложные утверждения\\
{\setlength{\jot}{0pt}
$\begin{aligned}
\{2\} &\in C, &\{2\} &\subset C\\
0     &\in C, &0     &\subset C\\
\varnothing &\in C, &\varnothing &\subset C\\
\{1,2\} &\in C, &\{1,2\} &\subset C\\
\{1,0\} &\in C, &\{1,0\} &\subset C\\
\{r\} &\in C, &\{r\} &\subset C\\
\{r,r\} &\in C, &\{r,r\} &\subset C\\
\{1,2,0\} &\in C, &\{1,2,0\} &\subset C\\
\{\{r\}\} &\in C, &\{\{r\}\} &\subset C
\end{aligned}$}

\bigskip\noindent \textbf{2.}
Пусть $B=\{1,2\}$\\
Перечислите все подмножества $B$.

\medskip\noindent \textbf{3.}
Пусть $R=\{e,b\}$.
Докажите от\\
противного, что $R \subset R$.

\medskip\noindent \textbf{4.}
Задайте явно множества\\
$G=\{s+f\mid s,f\in \mathbb{Z},s^2+f^2<9\}$\\
$P=\{3n+2\mid n\in \{0,1,2\}\}$\\
$X=\{\sqrt{c}\mid c\in \mathbb{N}\text{ и } c<5\}$

\medskip\noindent \textbf{5.}
Задайте через свойства множества\\
$\{4,9,16,25\}$,\,
$\{-1,0,1,2\}$,\,
$\{3,5,7,9\}$.

\medskip\noindent \textbf{6.}
Пусть $A=\{0,1\}$, $B=\{1,2\}$\\
Задайте явно множества\\
$L=\{e\mid e \in A \text{ и } e\notin B\}$\\
$V=\{K\mid K\subset A\}$\\
$J=\{y \in B\mid y<0\}$
\end{exercise}
\end{multicols}