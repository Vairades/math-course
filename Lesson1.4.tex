% !TeX root = ТеорияОтображений.tex
\section{Доказательства}
Напоминание кванторов
\begin{center}
\begin{tabular}{l l}
$\forall$ & -- для всякого\\
$\exists$ & -- существует/найдется \\
: & -- такой, что \\
$:=$ & -- равенство по определению \\
$\Rightarrow$ & -- следовательно (импликация)\\
$\iff$ & -- равносильно/тогда и только тогда.\\
$\overset{\mathrm{def}}{\iff}$ & -- равносильно по определению.
\end{tabular}
\end{center}
Важнейшим умением в математических доказательств является формулирование отрицаний, т.е. утверждений, опровергающий изначальное. Это умение нарабатывается с опытом, поэтому приведем только общие соображения. Следующие закономерности не являются универсальными и каждое отрицание должно появляться из логических соображений.

Если уверждение верно для всех объектов, то опровержением будет не выполнение даже для одного объекта. Напротив, если утверждение верно для какого-то объекта, то отрицанием будет не выполнение для всех объектов.

Отдельно необходимо сказать про импликацию двух утверждений $P\Rightarrow Q$. Ее отрицанием будет выполнение $P$ и \textbf{не} выполнение $Q$, и никак иначе. Любая другая ситуация не будет опровергать $P\Rightarrow Q$.

Теперь займемся практикой доказательств и работы с определениями. Ниже приведены несколько относительно простых утверждений и их доказательства. Доказательства намеренно сделаны претенциозно и со всей строгостью, чтобы у читателя появлялось представление того, как должно оформляться математическое доказательство. Но сначала дадим несколько определений.
\begin{definition}\label{def:segment}
Отрезком между числами $a$ и $b$, где $a,b\in \mathbb{R}$ и $a\leq b$, называется множество
\[[a,b] := \{x\in \mathbb{R}\mid a\leq x\leq b\}\]
\end{definition}
\begin{definition}
Множество $X\subset\mathbb{R}$ -- ограничено сверху, если
\[\exists m_0\in\mathbb{R}:\forall a\in X\quad a\leq m_0\]
ограничено снизу, если
\[\exists m_1\in\mathbb{R}:\forall a\in X\quad m_1\leq a\]
и просто ограничено, если 
\[\exists m_2\in\mathbb{R}:\forall a\in X\quad |a|\leq m_2\]
\end{definition}
\begin{statement}
отрезок -- ограниченное множество.
\end{statement}
\begin{proof}
По определению отрезка \ref{def:segment}
\[[a,b] := \{y\in \mathbb{R}\mid a\leq y\leq b\}\]
А доказать необходимо ограниченность, то есть
\[\exists s_0\in\mathbb{R}:\forall x\in [a,b]\quad |x|\leq s_0\]
Для этого достаточно явно предоставить $s_0\in\mathbb{R}$ и показать почему оно подходит.\\
Пусть \[s_0 =|a|+|b|\quad(\text{или }s_0 = \max\{|a|,|b|\})\]
Теперь почему для этого $s_0$ выполнено условие ограниченности. Пусть $x_1\in[a,b]$, тогда
\[x_1\in [a,b]\Rightarrow a\leq x_1\leq b\]
Из свойств модуля получаем что
\[b\leq|b|\leq|a|+|b|=s_0\Rightarrow b\leq s_0\]
\[a\geq-|a|\geq-(|a|+|b|)=-s_0\Rightarrow a\geq -s_0\]
Собираем все вместе
\[-s_0\leq x_1\leq s_0\Rightarrow |x_1|\leq s_0\]
Эти рассуждения справедливы для произвольного $x_1$, а значит все элементы отрезка ограничены числом $s_0$.
\end{proof}
\begin{statement}
Множество $[0,1]\cap [2,3]$ -- пусто.
\end{statement}
\begin{proof}
Пусть
\[M=[0,1]\cap [2,3]:=\{x\mid x\in[0,1],x\in[2,3]\}\]
Предположим противное, то есть $M\neq\varnothing$. Это значит что
\[\exists x_0\in M\]
Распишем продробнее
\[x_0\in M\Rightarrow
\begin{cases}
x_0\in [0,1]\\
x_0\in [2,3]
\end{cases}
\Rightarrow
\begin{cases}
0\leq x_0\leq 1\\
2\leq x_0\leq 3
\end{cases}
\Rightarrow
2\leq x_0\leq 1\\
\]
Это противоречие, значит $M$ -- пусто.
\end{proof}

\begin{statement}
$A\cup B=\varnothing \iff
\begin{cases}
A=\varnothing\\ 
B=\varnothing
\end{cases}$
\end{statement}
При доказательстве тождества двух утверждений $P\iff Q$, необходимо доказывать сразу две импликации: $P\Rightarrow Q$ и $P\Leftarrow Q$. Тут можно привести аналогию -- чтобы доказать равенство двух чисел $a=b$, мало доказать что $a\leq b$, необходимо доказывать и $a\geq b$.
\begin{proof}
Сначала докажем импликацию вправо ($\Rightarrow$), то есть докажем что
\[A\cup B=\varnothing \Rightarrow
\begin{cases}
A=\varnothing\\ 
B=\varnothing
\end{cases}\]
По определению $A\cup B := \{x\mid x\in A \text{ или } x\in B\}$. Докажем от противного, что, если $A\cup B=\varnothing$, то $A=\varnothing$ ($A\cup B=\varnothing\Rightarrow A=\varnothing$). Действительно, если $A\neq\varnothing$, то
\[\exists x_0\in A \Rightarrow \exists x_0\in A\cup B\Rightarrow A\cup B\neq\varnothing\]
Это противоречие, значит $A$ -- пусто. Аналогичные утверждения справедливы и для $B$, таким образом и $A$, и $B$ -- пусты. Импликация вправо доказана.

Теперь импликация влево ($\Leftarrow$), а именно
\[A\cup B=\varnothing \Leftarrow
\begin{cases}
A=\varnothing\\ 
B=\varnothing
\end{cases}\]
По условию $A=\varnothing$ и $B=\varnothing$. Предположим что $A\cup B\neq\varnothing$, тогда
\[\exists x_0\colon x_0\in A\cup B\Rightarrow x_0\in A\text{ или } x_0\in B\]
Но и $A$, и $B$ -- пусты, а это противоречие. Таким образом утверждение доказано.
\end{proof}
Поговорим про равенства множеств, множества равны, если равны их определения, формализуется это так:
\begin{definition}\label{def:seteq}
Множества $A$ и $B$ одинаковы (или равны), если
\[\forall x\ (x\in A\iff x\in B) \overset{\mathrm{def}}{\iff} A=B\]
Неформально можно сказать так, в $A$ и $B$ лежат одни и теже элементы.
\end{definition}
На практиче часто встречается ситуация, что с определением неудобно работать. Для таких случаев доказываются утверждения, именуемые критериями, они эквивалентны нашему определению, но часто имеют более удобные формулировки.
\begin{statement}
(Критерий равенства множеств)\\
\[A=B\iff (A\setminus B)\cup(B\setminus A)=\varnothing\]
\end{statement}
\begin{proof}
($\Rightarrow$)\\
Положим $A=B$, тогда $A\setminus B=B\setminus A=\varnothing$, а по утверждению 1.4.5. мы знаем что $\varnothing\cup\varnothing=\varnothing$. Таким образом $(A\setminus B)\cup(B\setminus A)=\varnothing$.

($\Leftarrow$)\\
Положим $(A\setminus B)\cup(B\setminus A)=\varnothing$, тогда по утверждению 1.4.5. получаем что
\[(A\setminus B)\cup(B\setminus A)=\varnothing
\Rightarrow
\begin{cases}
A\setminus B=\varnothing\\
B\setminus A=\varnothing
\end{cases}\]
Если $(A\setminus B)=\varnothing$, то $\forall x\ (x\in A\Rightarrow x\in B)$, поскольку иначе вычитание будет не пусто. Аналогичные выводы делаются и из $B\setminus A=\varnothing$, а именно $\forall x\ (x\in B\Rightarrow x\in A)$. Совмещая эти два утверждения получаем тождество
\[\begin{cases}
\forall x\ (x\in A\Rightarrow x\in B)\\
\forall x\ (x\in B\Rightarrow x\in A)
\end{cases}\iff
\forall x\ (x\in B\iff x\in A)\overset{\mathrm{def}}{\iff} A=B\]
\end{proof}
\begin{multicols}{2}
\begin{exercise}
\ \\
\noindent\textbf{1.}
Сформулируйте отрицание\\
ограниченности.

\medskip\noindent\textbf{2.}
Докажите утверждение 1.4.3.\\от противного.

\medskip\noindent\textbf{3.}
Докажите что $[0,1]\subset[0,2]$.

\medskip\noindent\textbf{4.}
Запишите отрицание\\
\[\forall \varepsilon>0\ \exists N\in\mathbb{N}\colon \frac{1}{N}<\varepsilon\]
\[\forall x\in\mathbb{Z}\ \exists y\in\mathbb{R}\colon y = \frac{1}{x}\]
\[\exists a\colon\forall n\in\mathbb{N}\ \frac{1}{n}<a\]
\[\exists M\colon\forall n\in\mathbb{N}\ n<M.\]

\medskip\noindent\textbf{5.}
Докажите или опровергните утверждения из предыдущего задания.

\medskip\noindent\textbf{6.}
Докажите из определения, что\\ $\{0,1\}\neq\{0,2\}$.
Затем проверьте это\\ неравенство по критерию.

\medskip\noindent\textbf{7.}
Пусть $A$ и $B$ -- произвольные множества.\\
Докажите $A\setminus B=\varnothing \iff A\subset B$.

\medskip\noindent\textbf{8.}
Пусть $A$ и $B$ -- множества. Докажите\\
$A\times B=\varnothing \iff
\left[
\begin{gathered}
A=\varnothing\\ 
B=\varnothing
\end{gathered}
\right.$

\medskip\noindent\textbf{9.}
Докажите что $\mathbb{N}\subset\mathbb{R}$ -- ограниченное множество.
\end{exercise}
\end{multicols}