% !TeX root = ТеорияОтображений.tex
\section{Доказательства}
Напоминание кванторов
\begin{center}
\begin{tabular}{l l}
$\forall$ & -- для всякого\\
$\exists$ & -- существует/найдется \\
: & -- такой, что \\
$:=$ & -- равенство по определению \\
$\Rightarrow$ & -- следовательно (импликация)\\
$\Leftrightarrow$ & -- равносильно/тогда и только тогда.
\end{tabular}
\end{center}
Теперь займемся практикой доказательств и работы с определениями. Ниже приведены два относительно простых утверждений и их доказательства. Доказательства намеренно сделаны претенциозно и со всей строгостью, чтобы у читателя появлялось представление того как должно оформляться математическое доказательство на примере не сложных понятий. Но сначала дадим несколько определений.
\begin{definition}\label{def:segment}
Отрезком между числами $a$ и $b$, где $a,b\in \mathbb{R}$ и $a\leq b$ называется множество
\[[a,b] := \{x\in \mathbb{R}\mid a\leq x\leq b\}\]
\end{definition}
\begin{definition}
Множество $X\subset\mathbb{R}$ -- ограничено сверху, если
\[\exists m_0\in\mathbb{R}:\forall a\in X\quad a\leq m_0\]
ограничено снизу, если
\[\exists m_1\in\mathbb{R}:\forall a\in X\quad m_1\leq a\]
и просто ограничено, если 
\[\exists m_2\in\mathbb{R}:\forall a\in X\quad |a|\leq m_2\]
\end{definition}
\begin{statement}
отрезок -- ограниченное множество.
\end{statement}
\begin{proof}
По определению отрезка \ref{def:segment}
\[[a,b] := \{y\in \mathbb{R}\mid a\leq y\leq b\}\]
А доказать необходимо ограниченность, то есть
\[\exists s_0\in\mathbb{R}:\forall x\in [a,b]\quad |x|\leq s_0\]
Для этого достаточно явно предоставить $s_0\in\mathbb{R}$ и показать почему оно подходит.\\
Пусть \[s_0 =|a|+|b|\quad(\text{или }s_0 = \max\{|a|,|b|\})\]
Теперь почему для этого $s_0$ выполнено условие ограниченности. Пусть $x_1\in[a,b]$, тогда
\[x_1\in [a,b]\Rightarrow a\leq x_1\leq b\]
Из свойств модуля получаем что
\[b\leq|b|\leq|a|+|b|=s_0\Rightarrow b\leq s_0\]
\[a\geq-|a|\geq-(|a|+|b|)=-s_0\Rightarrow a\geq -s_0\]
Собираем все вместе
\[-s_0\leq x_1\leq s_0\Rightarrow |x_1|\leq s_0\]
Эти рассуждения справедливы для произвольного $x_1$, а значит все элементы отрезка ограничены числом $s_0$.
\end{proof}
\begin{statement}
Множество $[0,1]\cap [2,3]$ -- пусто.
\end{statement}
\begin{proof}
Пусть
\[M=[0,1]\cap [2,3]:=\{x\mid x\in[0,1],x\in[2,3]\}\]
Предположим противное, то есть $M\neq\varnothing$. Это значит что
\[\exists x_0\in M\]
Распишем продробнее
\[x_0\in M\Rightarrow
\begin{cases}
x_0\in [0,1]\\
x_0\in [2,3]
\end{cases}
\Rightarrow
\begin{cases}
0\leq x_0\leq 1\\
2\leq x_0\leq 3
\end{cases}
\Rightarrow
2\leq x_0\leq 1\\
\]
Это противоречие, значит $M$ -- пусто.
\end{proof}

\begin{statement}
$A\cup B=\varnothing \Leftrightarrow 
\left[
\begin{gathered}
A=\varnothing\\ 
B=\varnothing
\end{gathered}
\right.$
\end{statement}
\begin{proof}
При доказательстве 
\end{proof}

\begin{multicols}{2}
\begin{exercise}
\ \\
\noindent\textbf{6.}
Сформулируйте отрицание\\
ограниченности.

\bigskip\noindent\textbf{6.}
Докажите утверждение 1.4.3.\\от противного.

\bigskip\noindent\textbf{6.}
Запишите отрицание\\
\[\forall n\in\mathbb{N}\ \exists m\colon m = 2n\]
\[\forall x\in\mathbb{Z}\ \exists y\in\mathbb{R}\colon y = \frac{1}{x}\]
\[\exists a\colon\forall n\in\mathbb{N}\ \frac{1}{n}<a\]
\[\exists M\colon\forall n\in\mathbb{N}\ n<M\]

\bigskip\noindent\textbf{7.}
Докажите или опровергните утверждения из предыдущего задания.

\bigskip\noindent\textbf{8.}
Докажите $A=B\Leftrightarrow (A\subset B\text{ и }B\subset A)$

\noindent\textbf{1.}
Пусть $A$ и $B$ -- множества. Докажите\\
$A\times B=\varnothing \Leftrightarrow 
\left[
\begin{gathered}
A=\varnothing\\ 
B=\varnothing
\end{gathered}
\right.\\
A\cap B=\varnothing \Leftrightarrow 
\begin{cases}
A=\varnothing\\ 
B=\varnothing
\end{cases}$

\end{exercise}
\end{multicols}
