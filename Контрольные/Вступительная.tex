\documentclass[12pt]{article}
\usepackage[margin=2cm]{geometry}
\usepackage[russian]{babel}
\usepackage{amsmath, amssymb, amsthm, mathtools, multicol}
\begin{document}
\section*{Диагностический тест (30 минут)}

\textbf{Важно:} цель теста --- понять, как вы рассуждаете. Объясняйте ответы словами.

\subsection*{Часть 1. Базовая логика}

\textbf{1.} Есть три утверждения:
\begin{itemize}
\item Все студенты любят математику.
\item Петя --- студент.
\item Петя любит математику.
\end{itemize}
Можно ли из первых двух сделать вывод о третьем? Почему?

\bigskip

\textbf{2.} 
\begin{itemize}
\item Все собаки умеют плавать.
\item Мурка умеет плавать.
\end{itemize}
Можно ли сказать, что Мурка --- собака? Объясните.

\bigskip

\textbf{3.} 
Если сегодня идёт дождь, то я беру зонт.  
Сегодня я взял зонт.  

Можно ли сделать вывод, что идёт дождь? Почему?

\bigskip

\textbf{4.}  
Если число делится на 4, то оно чётное.  
Дано четное число.  

Можно ли сказать, что оно делится на 4?

\bigskip

\subsection*{Часть 2. Отрицания}

Сформулируйте отрицания \textbf{словами}.

\textbf{5.} Все люди любят кофе.

\bigskip

\textbf{6.} Существует человек, который говорит по-японски.

\bigskip

\textbf{7.} Каждый студент понимает лекцию.

\bigskip

\textbf{8.} Найдётся число, которое больше 100.

\bigskip

\textbf{9.} Для любого числа можно найти большее.

\bigskip

\subsection*{Часть 3. Интуиция множеств}

Представьте, что множество --- это коробка с объектами.

\textbf{10.}  
$A = \{\text{яблоко, груша, апельсин}\}$  

Ответьте:
\begin{itemize}
\item лежит ли яблоко в коробке?
\item лежит ли банан?
\item сколько объектов в коробке?
\end{itemize}

\bigskip

\textbf{11.}  
$B = \{\{\text{яблоко}\}, \text{яблоко}\}$  

Ответьте:
\begin{itemize}
\item отличаются ли $\text{яблоко}$ и $\{\text{яблоко}\}$?
\item сколько объектов в коробке?
\end{itemize}

\bigskip

\textbf{12.}  
$C = \{1,2\}$, \quad $D = \{1,2,3\}$  

Ответьте:
\begin{itemize}
\item все ли объекты из $C$ лежат в $D$?
\item все ли объекты из $D$ лежат в $C$?
\end{itemize}

\bigskip

\textbf{13.}  
$E = \{1,\{1\}\}$  

Ответьте:
\begin{itemize}
\item лежит ли $1$ в $E$?
\item лежит ли $\{1\}$ в $E$?
\item лежит ли $\{1,1\}$ в $E$?
\end{itemize}

\bigskip

\subsection*{Часть 4. Логика пустоты}

\textbf{14.}  
В коробке нет ни одного предмета.

Можно ли сказать, что:
\begin{itemize}
\item все предметы в коробке красные?
\item существует предмет в коробке?
\end{itemize}
Объясните.

\bigskip

\textbf{15.}  
Можно ли сказать, что пустая коробка лежит внутри любой коробки?  
Попробуйте объяснить своими словами.

\bigskip

\subsection*{Часть 5. Задачи на рассуждение}

\textbf{16.}  
Есть утверждение:  
Если человек тренируется, то он становится сильнее.

Можно ли утверждать, что:
\begin{itemize}
\item если человек стал сильнее, то он тренировался?
\item если человек не тренируется, то он не станет сильнее?
\item могут быть люди, которые не тренеруются и стали сильнее
\item могут быть люди, которые тренеруются и не стали сильнее
\end{itemize}

\bigskip

\textbf{17.}  
Есть три утверждения:
\begin{itemize}
\item Все числа в коробке чётные.
\item В коробке есть число.
\item В коробке есть нечётное число.
\end{itemize}
Могут ли они быть одновременно верны?

\bigskip

\textbf{18.}  
Можно ли придумать пример, когда:
\begin{itemize}
\item верно: ``Если A, то B'';
\item но неверно: ``Если B, то A''?
\end{itemize}

\bigskip

\textbf{19.}  
Придумайте два разных объекта, которые выглядят похоже, но математически различаются  
(например, как $x$ и $\{x\}$). Объясните.

\bigskip

\textbf{20.}  
Попробуйте объяснить своими словами:
\begin{center}
Почему в математике важно быть точным?
\end{center}

\end{document}