\documentclass[a4paper,12pt]{article}

\usepackage[margin=2cm]{geometry}
\usepackage[T2A]{fontenc}
\usepackage[utf8]{inputenc}
\usepackage[english,russian]{babel}

\usepackage{amsmath, amssymb, amsthm, mathtools, multicol}
\usepackage{hyperref}
\usepackage{tikz,pgfplots}
\usepackage{venndiagram, wrapfig}

\begin{document}
\section{Контрольная №1. Принадлежность}
\textbf{1. (Определение)}\\
Сформулируйте определение множества.

\medskip
\noindent\textbf{2. (Определение)}\\
Сформулируйте определение пустого множества с использованием кванторов.

\medskip
\noindent\textbf{3. (Принадлежность)}\\
Пусть $A = \{\{2\}, 5, \{3,5\}\}$.  
Определите, какие утверждения истинны, а какие ложны:
\[
5 \in A,\quad
\{5\} \in A,\quad
\{2\} \in A,\quad
2 \in A,\quad
\{3,5\} \in A.
\]

\medskip
\noindent\textbf{4. (Пустое множество)}\\
Верно ли:
\[
\varnothing \in \varnothing, \qquad
\varnothing \in \mathbb{R} ?
\]
Обоснуйте ответ с опорой на определения.

\medskip
\noindent\textbf{5. (Работа с кванторами)}\\
Переведите на математический язык:
\begin{quote}
«Число 3 принадлежит множеству натуральных чисел, но не принадлежит множеству отрицательных чисел».
\end{quote}
\end{document}