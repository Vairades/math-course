\documentclass[12pt]{article}
\usepackage[margin=2cm]{geometry}
\usepackage[russian]{babel}
\usepackage{amsmath, amssymb, amsthm, mathtools, multicol}
\begin{document}
\section{Контрольная №2. Включение}
\textbf{1. (Определение)}\\
Сформулируйте \textbf{точное} определение подмножества $A \subset B$ в кванторах.

\medskip
\textbf{2. (Проверка включения)}\\
Пусть $X = \{1,2,3\}$, $Y = \{2,3\}$, $Z = \{1,3,4\}$.  
Определите, какие утверждения верны:
\[
Y \subset X,\quad
X \subset Z,\quad
\{2,4\} \subset Z,\quad
\varnothing \subset Y.
\]

\medskip
\textbf{3. (Перечисление подмножеств)}\\
Задайте множество всех подмножеств, множества $P=\{a,b\}$.

\medskip
\textbf{4. (Задание через свойства)}\\
Задайте через свойства множество чётных чисел, больших 0 и меньших 10.  
Запишите его в виде
\[
\{x\in\mathbb{N}\mid \dots\}
\]
и затем перечислите его элементы.

\medskip
\textbf{5. (Доказательство)}\\
Докажите, что $A \subset A$ для любого множества $A$.
\end{document}