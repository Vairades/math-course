\documentclass[a4paper,12pt,openany]{book}

\usepackage[margin=2cm]{geometry}
\usepackage[T2A]{fontenc}
\usepackage[utf8]{inputenc}
\usepackage[english,russian]{babel}

\usepackage{amsmath, amssymb, amsthm, mathtools, multicol}
\usepackage{hyperref}
\usepackage{tikz,pgfplots}
\usepackage{venndiagram, wrapfig}


\theoremstyle{definition}
\newtheorem{definition}{Определение}[section]

\theoremstyle{definition}
\newtheorem{statement}[definition]{Утверждение}

\theoremstyle{remark}
\newenvironment{commentary}
{\begin{quote}\itshape}
{\end{quote}}

\theoremstyle{definition}
\newtheorem*{example}{Примеры}

\newtheorem{exercise}{Упражнения}[chapter]

\frenchspacing
\pagestyle{plain}

\usepackage[dvipsnames]{xcolor}
%RoyalBlue — для определений и операторов
%ForestGreen — для ответов или утверждений
%BrickRed — для важных замечаний
%MidnightBlue — для теорем
%Sepia — мягкий тёплый акцент
%Purple — для вспомогательных обозначений
\definecolor{phos}{RGB}{0,178,178}

\title{Теория отображений для школьников}
\author{Черногорцев Михаил}
\date{11 февраля 2026 г.}

%\includeonly{Lesson1.4}

\begin{document}
\maketitle
\tableofcontents
% !TeX root = ТеорияОтображений.tex

\chapter{Теория множеств}

\begin{commentary}
Первой темой является вводный курс по теории множеств, в котором затронутся базовые термины и понятия. Очень важно освоить этот раздел, поскольку на протяжении всего курса мы будем работать именно с множествами.
\end{commentary}

\section{Принадлежность}

Само множество является неопределяемым понятием, но мы все еще можем дать интуитивное представление, поэтому
\begin{definition}\label{def:set}
Множество -- это набор произвольных элементов. Элементы в множестве не упорядочены и каждый элемент учитывается только один раз \textit{(даже если случились повторения)}.
\end{definition}
Точнее говоря, множество определяется через набор своих свойств и действий, которые вообще можно делать с ними (например объединять). По научному мы выбираем набор аксиом, то есть утверждений, которые принимаются за истину и \textit{чаще всего} интуитивно понятны, а все остальное выводится с помощью логики. Классически используется аксиоматика Цермелло-Френкеля, мы не будем ее разбирать поскольку это выходит за рамки курса, но важно понимать, что множество -- объект на поверхности очень простой, но коварный, и халатное обращение с множествами приводит к парадоксам.

Теперь подумайте, а как вообще можно задать множество? Самый простой ответ -- список, т.е. явно перечислить элементы
$$M=\{a,b,c\}\ B=\{4,M,\{\delta\}\}\ Y=\{0,1,\{x,1\}\}$$
Классически само множество обозначается заглавной буквой латинского алфавита, a элементы перечисляются внутри фигурныйх скобок через запятую. Определение \ref{def:set} не запрещает одному множеству быть элементом другого, образуя как бы пакет с пакетами. Более того множество может лежать само в себе. Отдельно отметим, что во всех приведенных выше множествах ровно три элемента ($\{x,1\}$ считается как один элемент).

Любой элемент должен либо лежать в множестве, либо нет, третьего не дано, и самое важное что мы хотим уметь делать -- это проверять лежит ли конкретный элемент в данном множестве.
\begin{definition}
Утверждение 'Элемент b принадлежит множеству M' на математическом языке записывается как $b \in M$. Отрицание этого утверждения как $b \notin M$, то есть b не принадлежит M.
\end{definition}
И, к сожалению, уже это бывает сложной задачей, поэтому рассмотрим 
\begin{example}
В следующем множестве 5 элементов и каждый выделен отдельным цветом 
$$ A = \{ \textcolor{cyan}{\{a,b\}},\textcolor{red}{-8},\textcolor{phos}{\{0,\{0\}\}},\textcolor{blue}{\{0\}},\textcolor{magenta}{\Gamma} \} $$
и только про эти 5 наборов символов можно сказать, что они принадлежат множеству $A$. Записи $\{a\}\in A$ или $0\in A$ будут неверны, в отличии от $\{a,b\}\in A$, $-8\in A$, $\{0,\{0\}\}\in A$, $\{0\}\in A$ и $\Gamma\in A$. Утверждение $\{2\}\in \{1,\{1,2\}\}$ так же неверно.
\end{example}
Еще одним важным понятием в математике является пустое множество, оно обозначается как $\varnothing$ (или просто $\{\}$), и оно так же определяется через свои свойства, а именно
\begin{definition}\label{def:varnothing}
Существует множество $\varnothing$ такое, что для всякого элемента $a$ справедливо, что $a \notin \varnothing$.
\end{definition}
Фраза 'для всякого элемента $a\ldots$'\ часто требует пояснений. Она \textbf{не} означает что мы берем конкретно букву $a$ латинского алфавита. Все ровно наоборот, так обозначается произвольный элемент, который мы вообще можем себе представить. Конкретные элементы будут обозначаться буквой с индексом: $x_0,\,a_3,\,b_1\ldots$ Словосочетание 'для всякого' принято заменять символом '$\forall$', такие символы называются кванторами и нужны для более короткой записи математических утверждений. Определение \ref{def:varnothing} записывается в кванторах как
\[
\exists\varnothing: \forall a\ (a \notin \varnothing)
\]
Другие частые обозначения:
\begin{center}
'$\exists$' -- существует/найдется \quad ':' -- такой, что \quad '$:=$' -- равенство по определению. 
\end{center}
Из определения \ref{def:varnothing} сразу можно понять, что если вы видите запись $x_0 \in \varnothing$ \textit{(вставьте любой набор символов вместо $x_0$)}, то это либо противоречие, либо парадокс, поскольку она явно противоречит определению.
\begin{exercise}
Проверьте следующие утверждения
\begin{multicols}{3}
\noindent \textbf{1.}
$B = \{1,\{1\},\{1,2\}\}$\\
$1 \in B$\\
$2 \in B$\\
$\{1\} \in B$\\
$\{2\} \in B$\\
$\{1,\{1\}\} \in B$\\
$\{1,2\} \in B$
\bigskip\\
\noindent \textbf{2.}
$D = \{\{0,\{1\}\},\{\{0\},\{1\}\}\}$\\
$0 \in D$\\
$\{0\} \in D$\\
$\{0,\{1\}\} \in D$\\
$\{\{0\},1\} \in D$\\
$\{\{0\},\{1\}\} \in D$\\
$\{1\} \in D$
\columnbreak
\bigskip\\
\noindent \textbf{3.}
$F = \{0,5,\{0\},\{2\},\{\}\}$\\
$\{0\} \in F$\\
$\{0,5\} \in F$\\
$\{0,2\} \in F$\\
$\{0,\{0\}\} \in F$\\
$\varnothing \in F$\\
$\{\{\},\varnothing\} \in F$
\bigskip\\
\noindent \textbf{4.}
$C = \{\{1\},\{2\},\{1,2\}\}$\\
$1 \in C$\\
$2 \in C$\\
$\{1,1\} \in C$\\
$\{2\} \in C$\\
$\{1,2\} \in C$\\
$\{\{1\},\{2\}\} \in C$
\columnbreak
\bigskip\\
\noindent \textbf{5.}
$G = \{1, a, \{1,a\}, \{\{1\}\}\}$\\
$1 \in G$\\
$a \in G$\\
$\{a\} \in G$\\
$\{1\} \in G$\\
$\{1,a\} \in G$\\
$\{\{1,a\}\} \in G$
\bigskip\\
\noindent \textbf{6.}
Переведите на русский\\
$\forall n\in\mathbb{N}\ \exists m\colon m = 2n$\\
$\forall x\in\mathbb{Z}\ \exists y\in\mathbb{R}\colon y = \dfrac{1}{x}$\\
$\forall n\in\mathbb{Z}\ \exists k\in\mathbb{N}\colon k>n$\\
$\exists a\colon\forall n\in\mathbb{N}\quad \frac{1}{n}<a$\\
$\exists M\colon\forall n\in\mathbb{N}\ n<M$\\
$\exists c\in\mathbb{Z}\colon\forall n\in\mathbb{Z}\ n\le c$
\end{multicols}
\end{exercise}
\begin{commentary}
С помощью $\varnothing$ в теории множеств задаются натуральные числа, например число 4 будет определено следующим образом 
$$\{\varnothing,\{\varnothing\},\{\varnothing,\{\varnothing\}\},\{\varnothing,\{\varnothing\},\{\varnothing,\{\varnothing\}\}\}\}$$
или для большей аутентичности
$$\{\{\},\{\{\}\},\{\{\},\{\{\}\}\},\{\{\},\{\{\}\},\{\{\},\{\{\}\}\}\}\}$$
\end{commentary}

% !TeX root = ТеорияОтображений.tex

\section{Включение}
Помимо принадлежности элемента к множеству, возникает необходимость рассматривать множество не целиком, а только его часть. Отсюда естественным образом возникает 
\begin{definition}\label{def:subset}
Множество A включено в множество B, если:\\
для всякого $x \in A$ верно что $x \in B$.
Обозначается как $A \subset B$. Множество $A$ называют подмножеством множества $B$. В кванторах $A\subset B \overset{\mathrm{def}}{\iff}\forall x\ (x\in A\Rightarrow x\in B)$.

Неформальная формулировка звучит так: Все элементы первого множества должны быть элементами второго множества.

Полезно заранее выписать отрицание утверждения $A \subset B$, а именно найдется такой элемент $x_0 \in A$, что $x_0 \notin B$. Записывается оно как $A \not\subset B$.
\end{definition}
\begin{example}
Пусть $A=\{1,2,\{0\}\}$, проверим утверждение $\{1,2\} \subset A$. В множестве $\{1,2\}$ всего два элемента и достаточно явно проверить определение.
\begin{center}
$1 \in \{1,2\}$ и $1 \in A$ -- верно\\
$2 \in \{1,2\}$ и $2 \in A$ -- верно
\end{center}
Значит утверждение $\{1,2\} \subset A$ -- верно. Теперь проверим $\{0,2\} \subset A$ аналогичным образом
\begin{center}
$2 \in \{2,0\}$ и $2 \in A$ -- верно\\
$0 \in \{2,0\}$ и $0 \in A$ -- неверно
\end{center}
Поскольку $0 \notin A$ вся логическая цепочка рушится и получается что $\{0,2\} \not\subset A$.
\end{example}
\begin{statement}
Для всякого множества $A$ справедливо что $\varnothing \subset A$.
\end{statement}
\begin{proof}
На этом утверждении удобно продемонстрировать структуру математического доказательства, а также пару частых приемов. Прежде всего доказательство будет от противного, его структура выглядит так:\\
Предполагаем что утверждение 1.2.2 неверно $\rightarrow$ с помощью рассуждений получаем противоречие $\rightarrow$ мы предположили что 1.2.2 неверно и получили ошибку, значит 1.2.2 верно.

Предположим противное. Теперь необходимо сформулировать отрицание утверждения 1.2.2, оно звучит так
\begin{center}
Нашлось такое множество $A$, что $\varnothing \not\subset A$ -- это то, что мы предполагаем
\end{center}
По предположению $\varnothing \not\subset A$, распишем это утверждение -- найдется $x_1 \in \varnothing$, такой что $x_1 \notin A$ (см. \ref{def:subset}). Тот кто внимательно читал предыдущий раздел уже увидели, что фраза "найдется $x_1 \in \varnothing$"\,не может быть правдой, поскольку противоречит определению \ref{def:varnothing}. Мы пришли к противоречию (получили ложное утверждение), таким образом 1.2.2 верно.
\end{proof}
До сих пор упоминался только один способ задать множество -- явно перечислить его элементы. Этот способ удобен при работе с \textit{маленькими} множествами и абсолютно не подходит для всего остального. На практике сначала берется уже готовое множество, и из него выбирается подмножество, отвечающее конкретному условию. Классический шаблон, котором мы будем пользользоваться:
\[M =\textcolor{blue}{\{}\text{\textcolor{phos}{натуральные числа }}\textcolor{BrickRed}{\text{со свойством}}\, \textcolor{Purple}{\text{меньше }6}\textcolor{blue}{\}}\]
Или же на математическом
\[M =\textcolor{blue}{\{}\textcolor{phos}{x\in \mathbb{N}}\ \textcolor{BrickRed}{|}\ \textcolor{Purple}{x<6}\textcolor{blue}{\}}\]
Разберем подробнее. \textcolor{blue}{Фигурные скобки} обозначают границы множества, внутри описаны элементы, вещи снаружи нам не интересны. Далее мы видим вертикальный \textcolor{BrickRed}{разделитель} и текст по обе стороны. 

Текст \textcolor{phos}{слева} указывает что будет являться элементами множества \textit{(числа, формулы, функции, другие множества\ldots)}. Здесь можно указать из какого множества будут браться элементы \textbf{или} какое преобразование нужно сделать с элементом. Но очень важно чтобы до или после '\textcolor{BrickRed}{|}' было указано откуда берутся элементы. В нашем примере элементы из $\mathbb{N}$. Обозначение элемента $x$ выбрано произвольно, тоже множество можно задать и так
\[M =\textcolor{blue}{\{}\textcolor{phos}{\xi\in \mathbb{N}}\ \textcolor{BrickRed}{|}\ \textcolor{Purple}{\xi <6}\textcolor{blue}{\}}=\textcolor{blue}{\{}\textcolor{phos}{t\in \mathbb{N}}\ \textcolor{BrickRed}{|}\ \textcolor{Purple}{t<6}\textcolor{blue}{\}}=\{1,2,3,4,5\}\]
В математике важны не столько сами символы, сколько их связь между собой и взаимное расположение. 

Текст \textcolor{Purple}{справа} задает свойства, которым должны отвечать элемнты, в нашем примере элементы должны быть меньше 6. Сам \textcolor{BrickRed}{разделитель} заменяет слова 'со свойством' или 'такой, что'. Таким образом утверждение выше должно читаться так:
\begin{center}
$M$ -- множество элементов $\xi$ из натуральных чисел, со свойством $\xi$ меньше 6.
\end{center}
\begin{example}
\ \\
$B = \{\dfrac{h}{3}\mid h\in\mathbb{N},h<4\}=\{\dfrac{1}{3},\dfrac{2}{3},1\}$ -- тут $h$ делится на 3 перед попаданием в $B$\\[2pt]
$F=\{x^2\mid x \in \mathbb{N}\}$ -- здесь берутся натуральные числа и возводятся в квадрат\\
$I=\{x\in \mathbb{R}\ |\ x \notin \mathbb{Q}\}$ -- множество иррациональных чисел

Озвучим как читается последнее утверждение
\begin{center}
$I$ -- множество элементов $x$ из $\mathbb{R}$ таких, что $x$ не является рациональным.\\
Или множество элементов из $\mathbb{R}$, которые не рациональны.
\end{center}

Не лишним будет привести примеры некорректного задания множеств и частые ошибки\\
$M=\{\mathbb{Z}\ |\ x<6\}$ -- потерян элемент, не понятно что будет лежать в множестве\\
$K=\{x \in \mathbb{N}\ |\ y<7\}$ -- нет связи между названием элемента и условием. В множестве лежат иксы, а условие наложено на $y$.\\
$B=\{x\mid x>9\}$ -- не хватает информации о том, чем является икс. Целым? Рациональным? Это вообще число?\\
$D=\{a \subset \mathbb{N}\ |\ a<4\}$ -- перепутаны знаки $\in$ и $\subset$, из за чего смысл полностью потерян.\\
$Z=\{z\in\mathbb{N}\mid 2x\}$ -- справа нет условия.
\end{example}
\begin{multicols}{2}
\begin{exercise}
\ \\
\noindent \textbf{1.}
$C = \{ 1, 2, \{1,2\}, 0, \{r\} \}$\\
Найдите ложные утверждения\\
{\setlength{\jot}{0pt}
$\begin{aligned}
\{2\} &\in C, &\{2\} &\subset C\\
0     &\in C, &0     &\subset C\\
\varnothing &\in C, &\varnothing &\subset C\\
\{1,2\} &\in C, &\{1,2\} &\subset C\\
\{1,0\} &\in C, &\{1,0\} &\subset C\\
\{r\} &\in C, &\{r\} &\subset C\\
\{r,r\} &\in C, &\{r,r\} &\subset C\\
\{1,2,0\} &\in C, &\{1,2,0\} &\subset C\\
\{\{r\}\} &\in C, &\{\{r\}\} &\subset C
\end{aligned}$}

\bigskip\noindent \textbf{2.}
Пусть $B=\{1,2\}$\\
Перечислите все подмножества $B$.

\medskip\noindent \textbf{3.}
Пусть $R=\{e,b\}$.
Докажите от\\
противного, что $R \subset R$.

\medskip\noindent \textbf{4.}
Задайте явно множества\\
$G=\{s+f\mid s,f\in \mathbb{Z},s^2+f^2<9\}$\\
$P=\{3n+2\mid n\in \{0,1,2\}\}$\\
$X=\{\sqrt{c}\mid c\in \mathbb{N}\text{ и } c<5\}$

\medskip\noindent \textbf{5.}
Задайте через свойства множества\\
$\{4,9,16,25\}$,\,
$\{-1,0,1,2\}$,\,
$\{3,5,7,9\}$.

\medskip\noindent \textbf{6.}
Пусть $A=\{0,1\}$, $B=\{1,2\}$\\
Задайте явно множества\\
$L=\{e\mid e \in A \text{ и } e\notin B\}$\\
$V=\{K\mid K\subset A\}$\\
$J=\{y \in B\mid y<0\}$
\end{exercise}
\end{multicols}
% !TeX root = ТеорияОтображений.tex
\section{Операции с множествами}
В прошлом разделе мы рассмотрели как задать множества через свойства, такой способ удобен, но может легко приводить к ошибкам, если не хуже. Рассмотрим пример наивного отношения к множествам, а именно парадокс Рассела. Я приведу две формулировки, школьную версию и строго математическую.

Школьная версия носит название парадокс списков. Положим мы хотим составить список всех книг, которые не содержат в своем тексте своего названия. Нужно ли в этот список вписать его название? Если да, то он будет нарушать свое же условие, если нет, то его придется вписать.

Теперь приведем математическую формулировку. Рассмотрим множество \[M=\{A\mid \varnothing \subset A\}\text{ -- множество множеств, вкючающих в себя пустое.}\]
Под условие подходит любое множество (см. 1.2.2), то есть для всякого множества $D$, $D \in M$. В результате получается так называемое множество всех множеств, теперь покажем к каким последствиям это приведет. Пусть 
\[K=\{F\in M\mid F \notin F\}\text{ -- множество множеств, не содержащих себя.}\]
\begin{commentary}
Если кому-то психологически тяжело воспринимать утверждения $B\in B$ и $B\notin B$, то воспринимайте это как книгу, цитирующую саму себя.
\end{commentary}
Верно ли что $K\in K$? Докажем от противного, предположим $K\notin K$, это попадает под условие множества $K\Rightarrow K\in K$. Противоречие, значит $K\in K$.\\
Но подождите, давайте теперь докажем что $K\notin K$, так же от противного. Положим противное, то есть $K\in K$, тогда условие $K$ не выполнено, ведь в $K$ лежат множества, \textbf{не} содержащие себя $\Rightarrow K\notin K$. Получается противоричивая ситуация, $K\in K$ и $K\notin K$, а такое невозможно.

В чем же проблема? Множество $M$ задано не корректно, в нем даже косвенно не указано откуда нужно брать элементы. Проще говоря наша аксиоматика не допускает существования множества всех множеств.

Обсудим операции с множествами. Для двух множеств $A$ и $B$ мы хотим иметь возможность объединить вместе все их элементы и собрать из них новое множество:
\[A\cup B := \{x\mid x\in A \text{ или } x\in B\}\text{ -- объединение}\]
Далее мы хотим уметь отобрать общие элементы для этих двух множеств, т.е. те, где они пересекаются:
\[A\cap B := \{x\mid x\in A \text{ и } x\in B\}\text{ -- пересечение}\]
А так же элементы, которые есть в первом, но не втором множестве:
\[A\setminus B := \{x\mid x\in A \text{ и } x\notin B\}\text{ -- вычитание}\]
Обратите внимание, что первые две операции симметричны, а третья -- нет. Эти операции удобно визуализировать диаграммами Венна, они интуитивно понятны и не требуют пояснений.
\begin{center}
\begin{venndiagram2sets}[showframe={false}]
\fillA \fillB
\end{venndiagram2sets}
\begin{venndiagram2sets}[showframe={false}]
\fillACapB
\end{venndiagram2sets}
\begin{venndiagram2sets}[showframe={false}]
\fillOnlyA
\end{venndiagram2sets}
\end{center}

Для последней операции нам понадобится новый объект под названием упорядоченная пара, она обозначается как $(p,q)$, где $p$ -- первый элемент, а $q$ -- второй. Мы не будем строго задавать этот объект и ограничемся интуитивным представлением.

Предподожим у нас есть два множества $A=\{0,1,7\}$ и $B=\{g,h\}$, и мы хотим составить таблицу
\begin{wrapfigure}{l}{0.15\textwidth}
\centering
\begin{tikzpicture}[scale=0.5]
% множества
\def\A{0,1,7}
\def\B{g,h}
% координатная сетка
\foreach \x [count=\i] in \A
  \node[below] at (\i,0.5) {\x};
\foreach \y [count=\j] in \B
  \node[left] at (0.5,\j) {\y};
% точки таблицы
\foreach \x [count=\i] in \A
  \foreach \y [count=\j] in \B
    \fill (\i,\j) circle (5pt);
\fill[Green] (2,1) circle (5pt);
\fill[Purple] (3,2) circle (5pt);
\end{tikzpicture}
\end{wrapfigure}
Каждая точка отвечает упорядоченной паре. Например, зеленая точка -- это пара $(1,g)$, фиолетовая -- $(7,h)$. Или можно явно записать таблицу со всеми парами вместо точек:
\[\begin{array}{c|ccc}
h & (0,h) & (1,h) & (7,h) \\
g & (0,g) & (1,g) & (7,g) \\
\hline
 & 0 & 1 & 7 \\
\end{array}\]
Явно множество всех таких пар будет задаваться следующим образом:
\[\{(0,g),(1,g),(7,g),(0,h),(1,h),(7,h)\}\]
Зададим его через свойства, но для этого нужно их сформулировать. Принцип такой, на первых позициях стоят только элементы $A$, а на вторых только элементы $B$.
\begin{definition}\label{def:Dprod}
Декартовым произведением множеств $A$ и $B$ называется множество упорядоченных пар $(p,q)$, со свойством $p\in A$ и $q\in B$.
\[A\times B := \{(p,q)\mid p\in A,q\in B\}\text{ -- декартово произведение}\]
Если $A$ и $B$ -- дискретные множества, то декартово произведение визуализируется через сетку или таблицу.
\end{definition}
\begin{example}
Рассмотрим результаты разных операций на примере $\{1,2\}$ и $\{1,3\}$.
\begin{multicols}{2}
\begin{center}
\noindent
$\{1,2\}\cup \{1,3\}=\{1,2,3\}$ \\
$\{1,2\}\cap \{1,3\}=\{1\}$ \\
$\{1,2\}\setminus \{1,3\}=\{2\}$ \\
$\{1,3\}\setminus \{1,2\}=\{3\}$
\end{center}
\columnbreak
\begin{center}
\noindent
\ \\
$\{1,2\}\times \{1,3\}=\{(1,1),(1,3),(2,1),(2,3)\}$\\
$\{1,3\}\times \{1,2\}=\{(1,1),(1,2),(3,1),(3,2)\}$
\end{center}
\end{multicols}
Здесь видно что вычитание и декартово произведение не симметричны, если поменять множества местами -- результат изменится, как, например, при разности чисел.
\end{example}

\begin{multicols}{2}
\begin{exercise}
\ \\
\noindent\textbf{1.}
Сформулируйте отрицание\\
$x_1\in A\cup B$, 
$x_2\in A\cap B$, 
$x_3\in A\setminus B$.

\medskip\noindent\textbf{2.}
Пусть $A$ и $B$ -- множества.
Докажите,\\ что $A\setminus A=\varnothing,\ (A\setminus B)\cup B=A,$

\medskip\noindent\textbf{3.}
Задайте явно и визуализируйте\\
$\{\alpha,\beta,\gamma\}\times\{1,2,3\},\ \{0,1\}^2,\\
\{1,2,3\}\times\{\alpha,\beta,\gamma\},\ \{0,1\}^3$.

\medskip\noindent\textbf{4.}
Задайте через свойства\\
$U$ -- множество четных чисел\\
$S$ -- множество нечетных чисел.

\medskip\noindent\textbf{5.}
Докажите что нет одновременно\\
четных и нечетных чисел\\
т.е. что $U\cap S$ пусто.

\medskip\noindent\textbf{6.}
Докажите что $0\notin S$ и $U\cup S=\mathbb{Z}$.

\medskip\noindent\textbf{7.}
Пусть $X=\{0,3,b,c\},Y=\{0,c\},\\Z=\{3,b,d\}$.
Задайте явно:\\
$\begin{array}{llll}
X\cup Y & X\setminus Z & Z\cup X\cup Y & (Y\cap X)\times Y\\
Y\cup Z & Y\setminus Z & X\cap Y\cap Z & Y\cap (X\times Y)\\
Y\cap Z & (X\setminus Z)\setminus Y & (Y\cup Z)\setminus X & (Y\cup X)\times Y\\
Z\cap X & X\setminus (Z\setminus Y) & Y\cup (Z\setminus X) & Y\cup (X\times Y).
\end{array}$

\medskip\noindent\textbf{8.}
Пусть $A$ и $B$ -- произвольные множества.\\ Докажите\\
$x\in A\Rightarrow x\in A\cup B$\\
$x\in A\cap B\Rightarrow x\in A$\\
Верно ли обратное? Приведите примеры.

\medskip\noindent\textbf{9.}
Схематично визуализируйте следующие множества: 
$\mathbb{N}\times\{0,1\},\{0,1\}\times\mathbb{N},\mathbb{N}^2,\mathbb{N}\times\mathbb{Z}$.
\end{exercise}
\end{multicols}

% !TeX root = ТеорияОтображений.tex
\section{Доказательства}
Напоминание кванторов
\begin{center}
\begin{tabular}{l l}
$\forall$ & -- для всякого\\
$\exists$ & -- существует/найдется \\
: & -- такой, что \\
$:=$ & -- равенство по определению \\
$\Rightarrow$ & -- следовательно (импликация)\\
$\iff$ & -- равносильно/тогда и только тогда.\\
$\overset{\mathrm{def}}{\iff}$ & -- равносильно по определению.
\end{tabular}
\end{center}
Важнейшим умением в математических доказательств является формулирование отрицаний, т.е. утверждений, опровергающий изначальное. Это умение нарабатывается с опытом, поэтому приведем только общие соображения. Следующие закономерности не являются универсальными и каждое отрицание должно появляться из логических соображений.

Если уверждение верно для всех объектов, то опровержением будет не выполнение даже для одного объекта. Напротив, если утверждение верно для какого-то объекта, то отрицанием будет не выполнение для всех объектов.

Отдельно необходимо сказать про импликацию двух утверждений $P\Rightarrow Q$. Ее отрицанием будет выполнение $P$ и \textbf{не} выполнение $Q$, и никак иначе. Любая другая ситуация не будет опровергать $P\Rightarrow Q$.

Теперь займемся практикой доказательств и работы с определениями. Ниже приведены несколько относительно простых утверждений и их доказательства. Доказательства намеренно сделаны претенциозно и со всей строгостью, чтобы у читателя появлялось представление того, как должно оформляться математическое доказательство. Но сначала дадим несколько определений.
\begin{definition}\label{def:segment}
Отрезком между числами $a$ и $b$, где $a,b\in \mathbb{R}$ и $a\leq b$, называется множество
\[[a,b] := \{x\in \mathbb{R}\mid a\leq x\leq b\}\]
\end{definition}
\begin{definition}
Множество $X\subset\mathbb{R}$ -- ограничено сверху, если
\[\exists m_0\in\mathbb{R}:\forall a\in X\quad a\leq m_0\]
ограничено снизу, если
\[\exists m_1\in\mathbb{R}:\forall a\in X\quad m_1\leq a\]
и просто ограничено, если 
\[\exists m_2\in\mathbb{R}:\forall a\in X\quad |a|\leq m_2\]
\end{definition}
\begin{statement}
отрезок -- ограниченное множество.
\end{statement}
\begin{proof}
По определению отрезка \ref{def:segment}
\[[a,b] := \{y\in \mathbb{R}\mid a\leq y\leq b\}\]
А доказать необходимо ограниченность, то есть
\[\exists s_0\in\mathbb{R}:\forall x\in [a,b]\quad |x|\leq s_0\]
Для этого достаточно явно предоставить $s_0\in\mathbb{R}$ и показать почему оно подходит.\\
Пусть \[s_0 =|a|+|b|\quad(\text{или }s_0 = \max\{|a|,|b|\})\]
Теперь почему для этого $s_0$ выполнено условие ограниченности. Пусть $x_1\in[a,b]$, тогда
\[x_1\in [a,b]\Rightarrow a\leq x_1\leq b\]
Из свойств модуля получаем что
\[b\leq|b|\leq|a|+|b|=s_0\Rightarrow b\leq s_0\]
\[a\geq-|a|\geq-(|a|+|b|)=-s_0\Rightarrow a\geq -s_0\]
Собираем все вместе
\[-s_0\leq x_1\leq s_0\Rightarrow |x_1|\leq s_0\]
Эти рассуждения справедливы для произвольного $x_1$, а значит все элементы отрезка ограничены числом $s_0$.
\end{proof}
\begin{statement}
Множество $[0,1]\cap [2,3]$ -- пусто.
\end{statement}
\begin{proof}
Пусть
\[M=[0,1]\cap [2,3]:=\{x\mid x\in[0,1],x\in[2,3]\}\]
Предположим противное, то есть $M\neq\varnothing$. Это значит что
\[\exists x_0\in M\]
Распишем продробнее
\[x_0\in M\Rightarrow
\begin{cases}
x_0\in [0,1]\\
x_0\in [2,3]
\end{cases}
\Rightarrow
\begin{cases}
0\leq x_0\leq 1\\
2\leq x_0\leq 3
\end{cases}
\Rightarrow
2\leq x_0\leq 1\\
\]
Это противоречие, значит $M$ -- пусто.
\end{proof}

\begin{statement}
$A\cup B=\varnothing \iff
\begin{cases}
A=\varnothing\\ 
B=\varnothing
\end{cases}$
\end{statement}
При доказательстве тождества двух утверждений $P\iff Q$, необходимо доказывать сразу две импликации: $P\Rightarrow Q$ и $P\Leftarrow Q$. Тут можно привести аналогию -- чтобы доказать равенство двух чисел $a=b$, мало доказать что $a\leq b$, необходимо доказывать и $a\geq b$.
\begin{proof}
Сначала докажем импликацию вправо ($\Rightarrow$), то есть докажем что
\[A\cup B=\varnothing \Rightarrow
\begin{cases}
A=\varnothing\\ 
B=\varnothing
\end{cases}\]
По определению $A\cup B := \{x\mid x\in A \text{ или } x\in B\}$. Докажем от противного, что, если $A\cup B=\varnothing$, то $A=\varnothing$ ($A\cup B=\varnothing\Rightarrow A=\varnothing$). Действительно, если $A\neq\varnothing$, то
\[\exists x_0\in A \Rightarrow \exists x_0\in A\cup B\Rightarrow A\cup B\neq\varnothing\]
Это противоречие, значит $A$ -- пусто. Аналогичные утверждения справедливы и для $B$, таким образом и $A$, и $B$ -- пусты. Импликация вправо доказана.

Теперь импликация влево ($\Leftarrow$), а именно
\[A\cup B=\varnothing \Leftarrow
\begin{cases}
A=\varnothing\\ 
B=\varnothing
\end{cases}\]
По условию $A=\varnothing$ и $B=\varnothing$. Предположим что $A\cup B\neq\varnothing$, тогда
\[\exists x_0\colon x_0\in A\cup B\Rightarrow x_0\in A\text{ или } x_0\in B\]
Но и $A$, и $B$ -- пусты, а это противоречие. Таким образом утверждение доказано.
\end{proof}
Поговорим про равенства множеств, множества равны, если равны их определения, формализуется это так:
\begin{definition}\label{def:seteq}
Множества $A$ и $B$ одинаковы (или равны), если
\[\forall x\ (x\in A\iff x\in B) \overset{\mathrm{def}}{\iff} A=B\]
Неформально можно сказать так, в $A$ и $B$ лежат одни и теже элементы.
\end{definition}
На практиче часто встречается ситуация, что с определением неудобно работать. Для таких случаев доказываются утверждения, именуемые критериями, они эквивалентны нашему определению, но часто имеют более удобные формулировки.
\begin{statement}
(Критерий равенства множеств)\\
\[A=B\iff (A\setminus B)\cup(B\setminus A)=\varnothing\]
\end{statement}
\begin{proof}
($\Rightarrow$)\\
Положим $A=B$, тогда $A\setminus B=B\setminus A=\varnothing$, а по утверждению 1.4.5. мы знаем что $\varnothing\cup\varnothing=\varnothing$. Таким образом $(A\setminus B)\cup(B\setminus A)=\varnothing$.

($\Leftarrow$)\\
Положим $(A\setminus B)\cup(B\setminus A)=\varnothing$, тогда по утверждению 1.4.5. получаем что
\[(A\setminus B)\cup(B\setminus A)=\varnothing
\Rightarrow
\begin{cases}
A\setminus B=\varnothing\\
B\setminus A=\varnothing
\end{cases}\]
Если $(A\setminus B)=\varnothing$, то $\forall x\ (x\in A\Rightarrow x\in B)$, поскольку иначе вычитание будет не пусто. Аналогичные выводы делаются и из $B\setminus A=\varnothing$, а именно $\forall x\ (x\in B\Rightarrow x\in A)$. Совмещая эти два утверждения получаем тождество
\[\begin{cases}
\forall x\ (x\in A\Rightarrow x\in B)\\
\forall x\ (x\in B\Rightarrow x\in A)
\end{cases}\iff
\forall x\ (x\in B\iff x\in A)\overset{\mathrm{def}}{\iff} A=B\]
\end{proof}
\begin{multicols}{2}
\begin{exercise}
\ \\
\noindent\textbf{1.}
Сформулируйте отрицание\\
ограниченности.

\medskip\noindent\textbf{2.}
Докажите утверждение 1.4.3.\\от противного.

\medskip\noindent\textbf{3.}
Докажите что $[0,1]\subset[0,2]$.

\medskip\noindent\textbf{4.}
Запишите отрицание\\
\[\forall \varepsilon>0\ \exists N\in\mathbb{N}\colon \frac{1}{N}<\varepsilon\]
\[\forall x\in\mathbb{Z}\ \exists y\in\mathbb{R}\colon y = \frac{1}{x}\]
\[\exists a\colon\forall n\in\mathbb{N}\ \frac{1}{n}<a\]
\[\exists M\colon\forall n\in\mathbb{N}\ n<M.\]

\medskip\noindent\textbf{5.}
Докажите или опровергните утверждения из предыдущего задания.

\medskip\noindent\textbf{6.}
Докажите из определения, что\\ $\{0,1\}\neq\{0,2\}$.
Затем проверьте это\\ неравенство по критерию.

\medskip\noindent\textbf{7.}
Пусть $A$ и $B$ -- произвольные множества.\\
Докажите $A\setminus B=\varnothing \iff A\subset B$.

\medskip\noindent\textbf{8.}
Пусть $A$ и $B$ -- множества. Докажите\\
$A\times B=\varnothing \iff
\left[
\begin{gathered}
A=\varnothing\\ 
B=\varnothing
\end{gathered}
\right.$

\medskip\noindent\textbf{9.}
Докажите что $\mathbb{N}\subset\mathbb{R}$ -- ограниченное множество.
\end{exercise}
\end{multicols}
% !TeX root = ТеорияОтображений.tex
\section{Задачи к Главе 1}

\begin{multicols}{2}
\bigskip\noindent\textbf{2.}
Пусть $A,B,C$ -- множества. Докажите\\
$A\times (B\cup C)=(A\times B)\cup (A\times C)$\\
$(A\cap B)\times C=(A\times C)\cap (B\times C)$.

\bigskip\noindent\textbf{3.}
Пусть $A$ и $B$ -- множества.\\ Докажите что
\begin{itemize}
\item $A\setminus B\subset A$
\item $A\cap B\subset A\cup B$
\end{itemize}

\bigskip\noindent\textbf{5.}
Пусть $A$ и $B$ -- множества.\\
Докажите что $(A\setminus B)\cap B=\varnothing$.

\bigskip\noindent\textbf{6.}
Пусть $A,B\subset X$. Докажите, что\\
$A\cap B=\varnothing\Leftrightarrow A\subset X\setminus B.$

\bigskip\noindent\textbf{7.}
Пусть $A,B,C$ -- множества. Докажите\\
$A\setminus (B\cup C)=(A\setminus B)\cap (A\setminus C)$\\
$A\setminus (B\cap C)=(A\setminus B)\cup (A\setminus C)$.

\bigskip\noindent\textbf{8.}
Пусть $M\subset\mathbb{R}$. Докажите что\\
$(\mathbb{R}\setminus M)\setminus M=\mathbb{R}\setminus M$.

\bigskip\noindent\textbf{9.}
Пусть $A,B,C,D$ -- множества.\\
Докажите или опровергните\\
$(A\times B)\cap (C\times D)=(A\cap C)\times (B\cap D)$\\
$(A\times B)\cup (C\times D)=(A\cup C)\times (B\cup D)$.

\bigskip\noindent\textbf{10.}
Пусть $A$ и $B$ -- множества. Докажите\\
$A\subset B \Leftrightarrow A\cup B=B$\\
$A\subset B \Leftrightarrow A\cap B=A$.

\bigskip\noindent\textbf{11.}
Пусть $A$ и $B$ -- множества. Докажите
$A\setminus (A\setminus B)=A\cap B$.

\bigskip\noindent\textbf{12.}
Пусть $A,B,C$ -- множества.\\
Верны ли следующие утверждения?\\
Если $A\subset B$, то $A\times C\subset B\times C$.\\
Если $A\times C=B\times C$ и $C\neq\varnothing$, то $A=B$.

\bigskip\noindent\textbf{13.}
Пусть $B$ -- фиксированное множество. Найдите все множества $A$, для которых\\
$A\cup B=A\cap B$.

\bigskip\noindent\textbf{14.}
Найдите все множества $A$ и $B$, такие что
$A\times B=B\times A$.

\bigskip\noindent\textbf{15.}
Пусть $B$ -- фиксированное множество. Найдите все множества $A$, для которых\\
$A\setminus B=A$.

\end{multicols}
% !TeX root = ТеорияОтображений.tex

\chapter{Отображения}
\section{Терминология}
\begin{definition}\label{def:map}
Отображением множества $A$ в множество $B$ называется сопоставление каждому элементу $A$, ровно одного элемента $B$.

Почти всегда одновременно речь будет идти про несколько разных отображений, поэтому принято давать им обозначения в виде букв, например  $f,\,g,\,\varphi,\,\Gamma$ и тд.

Фраза 'Отображение $f$ множества $A$ в множество $B$' на математическом языке записывается как $f:A\rightarrow B$.
\end{definition}
Если отображение $f$ сопоставляет элементу $d$ элемент $v$, то это можно обозначить двумя способами:
\[d\mapsto v\text{ или } v=f(d)\]
первый способ прост для понимания и удобен когда мы хотим сказать что $d$ переходит в $v$. Второй способ изпользует название отображения и удобен когда в рассуждениях фигурируют несколько отображений.

\bigskip
\begin{tikzpicture}

% сетка
\draw[gray!30] (-0.5,-0.5) grid (3.5,3.5);

% подписи осей
\foreach \x in {0,...,3}
  \node[below] at (\x,0) {\x};

\foreach \y in {0,...,3}
  \node[left] at (0,\y) {\y};

% точки
\foreach \x in {0,...,3}{
  \foreach \y in {0,...,3}{
    \ifnum\numexpr\x+\y\relax=3
      \fill[red] (\x,\y) circle (3pt);
    \else
      \fill[black] (\x,\y) circle (1.5pt);
    \fi
  }
}

\end{tikzpicture}

\bigskip
\begin{tikzpicture}
% множества
\def\A{a,b,c,d,e,f}
\def\B{0,1,2,3,4,5}
%сетка
\draw[gray!30] (-0.5,-0.5) grid (5.5,5.5);
%отображение
\foreach \x in {0,...,5}{
  \foreach \y in {0,...,5}{

    \pgfmathparse{abs(\x-2)}

    \ifnum\y=\pgfmathresult
      \fill[red] (\x,\y) circle (3pt);
    \else
      \fill[black] (\x,\y) circle (1pt);
    \fi
  }
}
%координаты
\foreach \x [count=\i] in \A
  \node[below, text height=1.5ex, text depth=.25ex] at (\i-1,-0.5) {\x};
\foreach \y [count=\j] in \B
  \node[left] at (-0.5,\j-1) {\y};
\end{tikzpicture}

\bigskip
\begin{tikzpicture}
\draw[fill=blue!10] (0,0) rectangle (3,2);
\draw[->] (-0.5,0) -- (3.5,0);
\draw[->] (0,-0.5) -- (0,2.5);
\end{tikzpicture}

\bigskip
\begin{tikzpicture}
\begin{axis}[axis lines=middle]
\addplot[only marks] coordinates {
(1,1) (1,2) (1,3)
(2,1) (2,2) (2,3)
};
\end{axis}
\end{tikzpicture}

\begin{tikzpicture}[scale=0.8]
%сетка
\draw[gray!40] (-0.5,-0.5) grid (5.5,5.5);
%точки
\foreach \x in {0,...,5}
  \foreach \y in {0,...,5}
    \fill (\x,\y) circle (1pt);
\end{tikzpicture}


определение, примеры, график, 
\end{document}
