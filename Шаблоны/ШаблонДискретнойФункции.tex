\documentclass{article}
\usepackage{tikz}
\usepackage{amsmath}

\begin{document}

\begin{tikzpicture}[xscale=1.5, yscale=1.5]

% --- 1. Настройки: множества и шаг сетки ---
\def\A{{0},{1},{7}}       % элементы множества A (столбцы)
\def\B{{g},{h}}            % элементы множества B (строки)
\def\step{1}               % расстояние между точками

% --- 2. Подписи столбцов (A) ---
\foreach \val [count=\i] in \A {
    \node[below] at (\i*\step,0) {\val};
}

% --- 3. Подписи строк (B) ---
\foreach \val [count=\j] in \B {
    \node[left] at (0,-\j*\step) {\val};
}

% --- 4. Отношение R (пары: столбец/строка по индексам) ---
% Пример: {столбец/строка}, нумерация с 1
\def\R{{1/1},{1/2},{2/2},{3/1}}  % соответствует: 0/g,0/h,1/h,7/g

% --- 5. Автоматическая расстановка точек ---
\foreach \pair in \R {
    % Разделяем индекс столбца и строки
    \pgfmathparse{\pair[0]} \let\ix=\pgfmathresult
    \pgfmathparse{\pair[1]} \let\jy=\pgfmathresult
    % Ставим точку
    \fill (\ix*\step, -\jy*\step) circle (3pt);
}

% --- 6. (опционально) сетка ---
\foreach \i in {1,...,3} {
    \draw[gray!30] (\i*\step,-\step) -- (\i*\step,-2*\step);
}
\foreach \j in {1,...,2} {
    \draw[gray!30] (\step,-\j*\step) -- (3*\step,-\j*\step);
}

\end{tikzpicture}

\end{document}
